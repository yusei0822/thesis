\chapter{序論}

\section{研究背景}

\subsection{はじめに}
高齢者施設は,現代社会の基盤となるシステムである.一方で,少子高齢化やそれに伴う介護士の不足は,医療サービス利用者にとって大きな問題である.これらの問題を解決するために,介護士の労働環境の改善や医療施設の拡充等が検討されている.医療現場は,一旦変更してしまうと容易に元に戻すことが難しい.しかも医療現場は非常に複雑であり,サービスの被提供者のプライベートや安全と密接に関連していることから,実験を行うこと自体が時間・コスト・安全の面から現実的ではない.このため,最新技術の導入等の実験を行い,それらの効果検証が出来る医療シミュレータの開発が急を要しているものの,医療現場はプライベートな空間であり,これまで現実データを十分に獲得することが出来ず,有用なシミュレータの構築が難しかった.しかし,今後の日本における医療の重要性を考えると,個人の特性や意志を持った主体として介護者、被介護者を取り扱い,それらの詳細な相互作用を取り入れたシミュレータの構築が必要であると考えられる.

\subsection{日本の抱える諸課題}

我が国では,世界に先駆けて少子高齢化が深刻化している.1950年時点で5%に満たなかった高齢化率(65歳以上人口割合)は,1985年には10.3%,2005年には20.2%と急速に上昇し,2015年は26.7%と過去最高となっている.将来においても,2060年まで一貫して高齢化率は上昇していくことが見込まれており,2060年時点では約2.5人に1人が65歳以上の高齢者となる見込みである\cite{ex_kousei_v1}.

\subsection{医療現場の諸課題}

\subsection{国の取り組み}

\subsection{医療技術の発達}

\subsection{医療技術導入における課題}

\section{目的}

我が国日本では、医療技術の研究が盛んに行われているのにも関わらず、それらの導入・浸透には至っていない.そこで本論文では、各医療機関がそれら技術の導入の意思決定につながるシミュレーションモデルを構築することを目的とする.シミュレーション上で技術の性能評価を行い,それによって技術の導入促進を目指す.

\section{本論文の構成}

1章では本論文の研究背景として,日本,医療界の諸課題について説明を行った.それに加えて課題の解決を目指す技術の紹介を行い,それらを導入する上での課題点を整理することで本論文の目的を示した.
2章では,提案手法についての説明をしている.3章では提案手法の数値実験により手法の検証を行う.4章では3章で得られた結果をまとめ,それらから得られる示唆についての考察を行い論文のまとめとする.
