\chapter{序論}

\section{研究背景}

高齢者施設は,現代社会の基盤となるシステムである.一方で,少子高齢化やそれに伴う介護士の不足は,医療サービス利用者にとって大きな問題である.これらの問題を解決するために,介護士の労働環境の改善や医療施設の拡充等が検討されている.医療現場は,一旦変更してしまうと容易に元に戻すことが難しい.しかも医療現場は非常に複雑であり,サービスの被提供者のプライベートや安全と密接に関連していることから,実験を行うこと自体が時間・コスト・安全の面から現実的ではない.このため,最新技術の導入等の実験を行い,それらの効果検証が出来る医療シミュレータの開発が急を要しているものの,医療現場はプライベートな空間であり,これまで現実データを十分に獲得することが出来ず,有用なシミュレータの構築が難しかった.しかし,今後の日本における医療の重要性を考えると,個人の特性や意志を持った主体として介護者、被介護者を取り扱い,それらの詳細な相互作用を取り入れたシミュレータの構築が必要であると考えられる.

\subsection{日本の抱える諸課題}

我が国では,世界に先駆けて少子高齢化が深刻化している.1950年時点で5%に満たなかった高齢化率(65歳以上人口割合)は,1985年には10.3%,2005年には20.2%と急速に上昇し,2015年は26.7%と過去最高となっている.将来においても,2060年まで一貫して高齢化率は上昇していくことが見込まれており,2060年時点では約2.5人に1人が65歳以上の高齢者となる見込みである \cite{ex_kousei_v1}.

\subsection{医療現場のマクロ課題}

% 高齢者施設が増加していること
% 介護士の数が不足していること
% 不足している原因として、離職率が問題になっていること
% 上記3つをファクト(厚生労働省とかの信用のあるデータ)として医療現場が問題をたくさん抱えていることを長々と(これ大事)説明したいです

\subsection{高齢者施設運営における課題}
我が国では,平成12年に医療保険制度に加えて,「介護保険制度」が導入された.介護保険の総費用は,平成12年から平成21年にかけて,3.6兆円から7.4兆円へ年10%を超える伸びで増大し,月間利用者も184万人(平成12年度)から344万人(平成19年度)と増加した.当初218万人だった要介護認定者が,現在数倍になっており,要介護高齢者の状態像で,認知症・廃用症候群・脳卒中の3大症状の対策が重要であると言われている.
今村ら\cite{nursing_management}によると,今後の施設運営で考慮すべき問題点は大きく8点挙げられている.

\begin{itemize}
 \item 「昭和ヒトケタ世代」から「第一次ベビーブーマー」へ移行した際の問題
 \item 認知症の増加の問題
 \item 単独世帯の増加の問題
 \item 都市部の高齢化の問題
 \item 看取り(死亡)の場所と体制の問題
 \item 介護のリハビリテーションの問題
 \item 医療と介護の連携の問題
 \item 介護従事者の確保の問題
\end{itemize}

以上のように,今後高齢者施設を適切に運営していくためには,すぐれた療養環境の提供と個別ケアが求めらていると言える.また,施設だけで完結するのではなく,地域や家族の協力のもと,適切な在宅ケアを提供していく必要がある.

\subsection{国の取り組み}

% 介護士不足を解消するために厚生労働省が行なっている取り組みをまとめたいです
% 何年までに何人にしたいみたいな目標があったはずなので、その文章を作りたいです
% その目標達成のために多分給与の底上げとか動労環境改善とかしてると思うので、その取り組みについてだらだらとまとめて欲しいです
% 結論として言いたいのは、「介護士の数そのものを増やすことも大事だが、一人ができる介護量を増やすことも同程度に必要であり、そのサポートとなる技術が発達して来ている」ということ。そして次のsubsectionにうつる

\subsection{医療技術の発達}

% 医療ロボット(物運ぶとかコミュニケーションとか)の発達や、生体情報をとって介護に生かす技術(お漏らししたらアラート出すオムツとか)多分医療に使うための技術がたくさんあるはずなので、それについてのまとめを行なっていただきたいです

これら技術には、高齢者自身の自立生活支援、高齢者介護の支援、生活の質(QOL)と快適性の高揚とその維持など、高齢者に対する生活支援分野においては、その使用者が被介護者か介護者により自立支援技術と介護支援技術に分けられる。\\
・自立支援技術 \\
排泄、入浴、調理、食事、就寝・起床、洗濯、清掃、義肢・装具、移乗、生活圏移動 \\
・介護支援技術 \\
排泄、入浴、清拭、褥瘡予防、食事、就寝・起床、移載・移動、監視 \\
また、これら技術には開発の優先順位が設けられている。以下の表\ref{tech_classiication}では、高齢者を自立意識が高い、低い、認知症の3パターンで分類したものと、被介護者の状態として全介護、半介護、身体異常の3パターンで分類したものとのマトリクスを示している。

\begin{table}[htb]
  \caption[介護技術の分類]{介護技術の分類}
  \label{tech_classification}
  \centering
  \begin{tabular}{r|c|c|c}
     & 全介護 & 半介護 & 身体異常 \\ \hline
    自立意識強い & 自立支援 & \quad & \quad \\
    自立意識弱い & \quad & \quad &  \quad \\
    認知症 & \quad & 介護支援 & \quad \\
    \end{tabular}
\end{table}

・自立支援 \\
自立意欲が高いにもかかわらず自立できていないことに対する支援機器が必要 \\
・介護支援 \\
一部介護を要する自立意欲の弱い高齢者には肉体的・精神的・時間的に大きな負担 \\
表\ref{tech_classification}にあるように,全介護の状態でありながらも,自立意識の高い被介護者と,半介護の状態でありながら,自立意識が弱かったり,認知症である場合に,それぞれ自立支援機器と介護支援機器の開発が必要とされる.

\subsection{医療技術導入における課題}

しかし、前述の支援機器の実用化には以下のように多くの問題点が存在する。

\begin{itemize}
 \item 性能不十分
 \item 操作複雑
 \item 寸法・重量
 \item 高価
 \item 危険
 \item 公害
 \item プライバシーの侵害
\end{itemize}

これら問題により、新技術の導入は医療サービス被提供者にとってはもちろん、医療機関にとって簡単に行うことができない。これら技術の導入が進んで来なかった背景として、そもそも技術として不完全であることに加え、現場の忙しい看護師や医師たちが簡単に使えるようなものでなければいけないことなど多くの制約があり、中でも、導入した際の効果検証ができないことが、意思決定の大きなボトルネックとなっている。また、病院は収益が順調に出ている状態だと、リスクをとって環境を改善するインセンティブが湧きづらく、こういった技術導入へのインセンティブが働かないことも大きな課題として挙げられる。

以上のように,最新技術の導入等の実験を行い,それらの効果検証が出来る医療シミュレータの開発が急を要しているものの,医療現場はプライベートな空間であり,これまで現実データを十分に獲得することが出来ず,有用なシミュレータの構築が難しかった.しかし,今後の日本における医療の重要性を考えると,個人の特性や意志を持った主体として介護者、被介護者を取り扱い,それらの詳細な相互作用を取り入れたシミュレータの構築が必要であると考えられる.

シミュレーションを行う際に気を付けなければならないのは,シミュレーションで用いる行動ルールのパラメータの妥当性とその客観性である.実際の介護の動きの計測結果から客観的に抽出されるルールやパラメータを入力とするシミュレーションが理想的である.しかし現在のところ,行動ルールのパラメータを抽出することを可能にするほどの精度の高い人流計測をするための研究はあまりされていない.これは一つには現在主流である単純な画像解析による手法の限界,もう一つには介護という環境がプライベートな空間であり,そもそも計測をできるような環境にない,という事が挙げられる.数値シミュレーションの現状を見ると,ポテンシャルモデルを用いた手法1),セルオートマトンを用いた手法2),個別要素法に基づく手法3),4),5)など様々な手法が存在している.これらの手法は計算の単純化による計算効率の高さや簡単な追従行動の再現能力などの長所を持っている.しかし,ポテンシャルモデルでは設定されるポテンシャルの客観性の低さ,通路閉塞のような非線形現象の予測の困難さ,といった問題がある.また,セルオートマトンを用いた手法ではルールの妥当性・客観性に問題がある.個別要素法に基づく手法は,人が密集する場合まで扱えるという利点はあるが,ここでもやはりルールを客観的に決定することが難しい.

\section{目的}

我が国日本では、医療技術の研究が盛んに行われているのにも関わらず、それらの導入・浸透には至っていない.そこで本論文では、各医療機関がそれら技術の導入の意思決定につながるシミュレーションモデルを構築することを目的とする.本研究の第一ステップとして,現在医療の現場で大きな課題となっている排泄介助にスコープを当て,排泄介助のシミュレーション上で技術の性能評価を行い,それによって技術の導入促進の意思決定に資するプラットフォームの構築をおこなう.

\section{本論文の構成}

1章では本論文の研究背景として,日本,医療界の諸課題について説明を行った.それに加えて課題の解決を目指す技術の紹介を行い,それらを導入する上での課題点を整理することで本論文の目的を示した.
2章では,提案手法についての説明をしている.3章では提案手法の数値実験により手法の検証を行う.4章では3章で得られた結果をまとめ,それらから得られる示唆についての考察を行い論文のまとめとする.
