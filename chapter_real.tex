\chapter{序論}

\section{研究背景}
日本では1960年代以降モータリゼーションが進み、日本の自動車保有台数は50年前と比べて約5.5倍となる約8100万台にまで増加し、自動車は現代の日常生活には欠かせない存在となった.一方で、自動車の増加に伴い自動車に関する様々な交通問題が生じてきた.中でも渋滞や事故は多大なる人的・経済的な損失を生み出すため、それらを防止するための施策がこれまでにも多く実行されてきた.最近では危険運転や高齢者ドライバーの判断ミスなど人為的要因による事故・トラブルが散見されるようになった.さらにEV車やカーシェアリング、自動運転といった新システムの実用化に向けた検証や自動車が環境に与える影響評価などの必要性が高まっており、交通問題の多様化が進んでいる.

これらの問題・課題を解決するためにたびたび交通流シミュレーションが用いられる.近年では問題が大規模化・複雑化しているため、現実の交通網を用いた検証が困難になりシミュレーションの有用性は高まっている.それに伴った交通流シミュレーションの技術的な進歩も見られる一方で、シミュレーションの信頼性に関する問題と実行時間に関する問題は常に存在している.シミュレーションの信頼性に関しては交通流シミュレーションは物理シミュレーションのように厳密解が存在しないことや交通という現象自体が非常に複雑な挙動を示すこと、十分なデータの用意が困難であることなどから、シミュレーションの信頼性の評価が難しい傾向にある.シミュレーション実行時間に関しては問題の大規模化に伴い計算時間が指数関数的に増加する場合に、現実的な時間でシミュレーション実行ができなくなる可能性がある.

本論文では阿部らによる交通量に着目してシミュレーションの信頼性の評価を試みた先行研究を元にして、シミュレーションの高速化を試みる.それにより大規模シミュレーションであっても信頼性を担保しながら現実的な時間で実行可能になることを目指す.

\section{交通流シミュレーション}

\subsection{交通流シミュレーションの対象となる問題}
近年の交通問題は、現実の交通網を用いた実験の実施が困難な大規模かつ複雑な問題が多いため、問題解決のために交通流シミュレーションを使用する機会は増加傾向にある.具体的に交通流シミュレーションを用いた研究事例として渋滞や事故の解消・緩和\cite{ex_jam,ex_accident}、環境汚染の影響評価\cite{ex_environment}、カーシェアリングの持続可能性評価\cite{ex_car_sharing}などが挙げられる.一方で実務的に交通流シミュレーションがよく利用されるケースはシナリオ分析である.これは分析対象となる状況に対して、ある施策を講じた際にどのような交通流になるかを確認し、その施策の評価を行うことである.具体的にシナリオ分析を行うケースとして、突発的なイベントが開催され一時的に急激な交通量の増加が見込まれる際の輸送計画の施策の決定、急速な開発による周辺エリアの恒常的な交通量の増加に伴う交通渋滞の緩和の施策の決定などが挙げられる.前者のケースでは、日常的にイベントなどが行われる訳ではなく周辺道路がイベントの開催を受け入れるだけの交通容量や知見がない場合や、ある時間までに必要な貨物や人員等の輸送を完了させなければならないといった制約条件が課される場合があり、現実の交通網を用いた実験はほぼ不可能である.後者のケースでは、結果の予想ができないままに現実の交通網で施策を実行すれば周辺住民の反対を買うことになり、その上施策の実行に多大な費用がかかるため費用対効果の面からも現実の交通網を用いた実験は適切ではない.このような背景からシナリオ分析の際に交通流シミュレーションが用いられることが多い.本研究ではイベント開催におけるシナリオ分析に着目している.

\subsection{交通流シミュレーションの種類}
交通流シミュレーションを用いる際には各問題に適したモデルが構築されるが、それらはマクロモデルとミクロモデルの2種類のモデルに大別される.その特徴を以下の表\ref{model_explaination}に示す.

\begin{table}[htb]
  \caption[交通流シミュレーションのモデル比較]{交通流シミュレーションのモデル比較}
  \label{model_explaination}
  \centering
  \begin{tabular}{r|c|c}
     & マクロモデル & ミクロモデル \\ \hline
    追跡対象 & 車両群 & 車両一台 \\
    計算コスト & 小さい & 大きい \\
    出力データ & 統計量 & 車両ごとのデータ
  \end{tabular}
\end{table}

マクロモデルではまとまった車両群の挙動を流体として扱うため、状態変数として車両密度や交通量といった統計量を用いることが多い.そのため計算コストが抑えられ、広域でのシミュレーションが実施しやすい長所がある一方で、車両ごとの挙動を把握できないため複雑なネットワークにおける詳細な交通状況を得たい場合には不向きである.ミクロモデルでは個別の車両の挙動を扱うため、車両ごとに詳細な挙動を制御することが可能であり、その各車両の挙動によって全体としての交通状況が創発される.そのため細かなスケールであっても、可視化により詳細な交通状況を把握することが可能であることや、車両ごとに詳細な設定を行い複雑なシナリオに対しても施策の評価が可能であることなどが長所として挙げられる.一方で計算コストが大きいため現実的な時間でシミュレーションできる範囲が限られるという課題がある.本研究ではイベント開催におけるシナリオ分析に着目し、イベント開催場所付近の詳細な状況を把握できることや特定車両の挙動を追跡可能であることなどからミクロモデルを採用する.

\subsection{交通流シミュレーションの使用における手順}

シナリオ分析でシミュレーションを活用する際には、はじめにシナリオ分析の対象となる現状の交通状況を再現するケースを作成する.このフェーズではシミュレーションの信頼性を担保し、シミュレーション結果を保証することが重要である.その上で作成した現状のケースに対して様々な施策を講じて交通状況を確認し、施策の評価をする.このフェーズでは前フェーズで作成したケースの一部の入力データを施策に基づき変化させることにより、施策を実施した際の交通状況を再現している.
このうち、シミュレーションを活用する上で最も重要な点はシミュレーションの信頼性の担保である.

シミュレーションの信頼性を担保するものはシミュレーションモデルの正確性と入力データの正確性である.モデルの正確性が担保されているシミュレーターを用いる場合には入力データの正確性の保証がシミュレーションの信頼性の保証になる.したがって、信頼性が担保されるように現状の交通状況を再現するケースを作成するためには、正確性が保証されている入力データが求められる.シミュレーションを実行するにあたり必要な入力データのうち、ネットワーク・信号現示・制限速度の情報などは直接観測可能であるが交通需要は直接観測不可能である.交通需要とはある地点からある地点へ移動する需要を意味しており、シミュレーション上では車両の移動経路や台数などに関わる必須の入力データである.そのため、交通需要は観測可能な他の情報から推定する必要がある.交通需要の表現方法の一つにOrigin-Destination表(OD表)があり、以下の表\ref{od_table}のように表される.OD表ではどの地点からどの地点まで何台の車両が移動するかを表しており、OD表を推定する手法(OD推定手法)はこれまでにも多く開発されている.OD推定手法の既存研究については1.3章で述べる.

\begin{table}[htb]
  \caption[OD表の例]{OD表の例(Veh/h)}
  \label{od_table}
  \centering
  \begin{tabular}{r|c|c|c}
    \diagbox{目的地}{出発地} & 出発地A & 出発地B & \cdots \\ \hline
    目的地a & 100 & 200 & \cdots \\ \hline
    目的地b & 400 & 300 & \cdots \\ \hline
    \vdots & \vdots & \vdots & \ddots
  \end{tabular}
\end{table}

\section{OD推定手法の既存研究}

OD推定手法は、OD表の用途や推定に用いるモデルの違いによって、多くの手法が提案されている.ここでは推定に用いるデータ取得のタイミングにより静的推定と動的推定に分類する.

\subsection{動的推定}

動的推定とはリアルタイムに交通データを取得し、時間発展を有するモデルに次々にデータを当てはめて状態を修正することで、動的な推定が行えるようにし
たものである.モデル内の潜在的な状態変数$x$をOD表とし、システムから観測可能な変数$y$を現実における観測可能な量(交通量など)としたときに、隠れマルコフモデル等に代表される状態空間モデルによって$x$から$y$が算出される.一方$y$は実際に観測可能であるため、状態空間モデルが既知であればモデルから出力される再現値と現実の観測値とを比較することによって$x$を逆算的に推定することが可能である.これを各時刻で連続的に行う手法が動的推定に分類される.この手法は一般にデータ同化技術として扱われ、具体的にはカルマンフィルタを用いたもの[10]などが該当する.前述の通り、隠れマルコフモデルの文脈ではOD表が隠れ変数として扱われるため、この手法はOD表そのものを推定するための手法というよりも、システム全体の状態を補正するための手法と捉えるべきである.したがって、シナリオベースのシミュレーションを行う際の入力データの推定とは用途が異なるといえる.

\subsection{静的推定}

静的推定とは、調査データを一度収集・保存し、後処理的にOD表を推定するものである.静的推定は0D表そのものを使用するために用いられるため、数多くの手法が研究されており、更に細かな分類を考えることができる.
ここでは、トリップベースのモデル、リンク交通量最適化モデル、数理計画モデルの3つに細分し以下詳細を説明する.

\subsubsection{トリップベースのモデル}

トリッブベースのモデルでは、人口分布、移動目的、需要量等を調査から求め、そこからトリップを作成し、それを集計してOD表を作成する組み立て的手法である.トリップとは移動目的を持った出発地・目的地間の一方向の移動を定量化する移動単位であり、ODベアやOD交通最と直接対応するものであるため、直接OD表へ変換することが可能である.中長期的なインフラ設計やそのための交通需要予測を目的として、四段階推定法[11]の1ブロセスとして1950年代から使用されているが、OD表から道路ネットワークへの交通量配分が陽に行われないため、推定結果から求まる交通量と現実の交通量との整合性は担保されないという欠点が存在する.

\subsubsection{リンク交通量最適化モデル}

トリップベースにおける欠点であった、現実の交通量との整合性を担保させて精度を上げるために、リンク交通量最適化モデルが提案された.リンク交通量とは定点観測で得られる道路上の交通量であり、現実に観測されたリンク交通量(観測リンク交通量)とOD推定モデル上で再現されるリンク交通量(再現リンク交通量)とが合致するように最適化を行う手法がリンク交通量最適化モデルである.このモデルではOD表からリンク交通最への変換、すなわち道路ネットワークへの交通量配分を規定した上で、OD表を変数としてリンク交通量の最適化が行われる.この道路ネットワークへの交通量配分には、所与の経路候補から確率的に経路を選択するロジットモデルを仮定したものがしばしば利用される.ロジットモデルとは一般に、選択肢$i$における効用$V_i$を与えたときに効用の高い経路が高い確率で選択されるというモデルである.特に経路選択上しばしば現れる、複数の選択肢(個別の経路に談当する)から確率を求める条件下では、多項ロジットモデルが用いられ、以下のような式で表現される.

\begin{equation}
  x^2 - 6x + 1 = 0
\end{equation}

ロジットモデルでは、ある選択肢の効用が突出して大きければ確率が1に近づき、すべての選択肢の効用が等しければ確率が等分されるこの振る舞いが現実における価俺判断との親和性が高いことから、経路選択モデルとして取り入れられているロジットモデルを仮定した経路選択では、車両の通過するリンクが確率的に表されるためそれを考慮した最適化手法を導出することができ、具体的には以下のものが提案されている.

・エントロビ一最大化モデル[12、 13、14]
・最尤推定モデル[15、16]
ㆍ GLS (Generalized Least Square) モデル[17]
・ベイズ推論モデル[18]

エントロビー最大化モデルは、事前に0D表の分布確率を与え、観測リンク交通量の出現確率が最大となるOD表を算出するモデルである、事前に与えるOD表はトリップベースのモデル等で作成することができ、本手法はその精度を向上させるためのものと捉えることができる近年では、プローブデータを活用して交通量の再現性を向上させるといった事例(19]が存在する.最尤推定モデルとは、エントロピー最大化モデルと類似するもので、事前に与えられる0D表が定常的であり、分布ではなく一つの値に確定している、というものである.その後の最適化手法はエントロピー最大化モデルに準じる.GLSモデルとは、視測リンク交通量と再現リンク交通量の差、初期に与えるOD表と推定で求められる0D表との差、このそれぞれを、予測誤差を考慮した上で最小化するモデルである、リンク交通量と0D表の観測値・再現値間の差として12ノルムを用いると、解くべき間題は一艘化最小二更法で表すことができるためGLSモデルと呼ばれる.ベイズ推論モデルとは、確率論的要薬の取り扱いにベイズ推の考えを用いて、確率分布を更新して解を求める方法である.

\subsubsection{数理計画モデル}

また、ロジットモデルを仮定しない一般的なモデルとして、数理計画法に基づくモデルが存在する.2レベル計画問題としての定式化はその一例である(図13).これは0D推定問題を、OD表の更新問題と0D表から道路ネットワークへの交通量配分問題の2つの小問題に分けそれらを交互に解くものであり、収束時の0D表が推定値となる.また、これはGモデルの類型とみなせることから、モデル化の詳細はロジットモデルを仮定したものと共通する点も多い、

配分関数fに対して0D表に関する微分が求まれば、OD表の単位変化量あたりの配分関数の変化量が求まり、そこから更新関数gを定義することができる.したがってこのモデルでは、微分可能な関数表現ができる任意の配分方法を探用可能である.適用例として、配分方法に利用者均衡配分を用いたもの[20]などが存在する.こうしたオフライン推定に関する研究として、交通量に基づくものはBera and Raoが詳細にまとめている[21].

\subsection{阿部らの手法}

阿部らは

\subsubsection{問題設定}

\subsubsection{定式化}

\subsubsection{二次計画問題}

\subsubsection{手法の概要}

\subsubsection{手法の課題点}

\section{目的}

\section{本論文の構成}

\chapter{提案手法}

\section{手法の概要}

\section{グループ化処理}

\subsection{経路によるグループ化処理}

\subsection{距離によるグループ化処理}

\chapter{数値実験}

\section{実験条件}

\section{対象ネットワーク}

\subsection{架空ネットワーク}

\subsection{現実ネットワーク}

\section{評価指標}

\section{結果}

\subsection{グループ生成}

\subsection{計算時間}

\subsection{リンク交通量比較}

\chapter{結論}

\section{本研究のまとめ}

\section{今後の展望}
