\chapter{謝辞}

多くの方からのご指導,ご協力を頂き,本論文の完成まで至ることが出来たと思い,この場をお借りしてお礼を申し上げます.

指導教員であり,本研究の機会を与えていただいた,吉村教授に感謝いたします.研究への姿勢や考え方など,右も左もわからなかった私に一から丁寧に教えていただきました.また,就職活動でもお気遣いいただき,本当に感謝しております.研究を通じて先生に教えていただいた,物事を構造的に捉え,仮説を持って取り組む姿勢は,授業では決して得られない貴重な学びになりました.先生から学んだ仮説思考とも言える考え方は,研究の領域に留まらず,人生においてとても重要なものであると確信しております.今回私は,これまで研究室にはなかった医療というフィールドでの研究を0から始めたわけですが,このようなチャレンジングな機会を与えていただいたこと,また研究内容が社会的意義のあるものだと感じたまま研究に取り組めたことは自分の修士課程にとってとても大きなことでした.本研究での経験や学びを忘れず,社会のために生かして参りたいと存じます.ゼミのディスカッションや研究打ち合わせでは,私に合わせて理解のしやすい説明も織り交ぜながら,医療現場について,研究発表について,教えていただきました.先生とのミーティングは,医療業界全体が抱えるマクロ的な課題をはじめとして,私の母親の例だと,といったようにミクロ的な物事の捉えかた,縦横無尽に課題を探索していく思考の柔軟さに圧倒されることが多く,非常に学びの多い時間でした.とても忙しいはずなのに何冊も本を読み,貪欲に知識を蓄えていく姿勢は,私も今後死ぬまで持ち続けていきたいと心から感じております.これからは,吉村研究室の名に恥じない,社会に価値を提供できる人間になり,吉村研究室の卒業生として世界に名を馳せるよう精進いたします.また,いつも単位のことを気にかけてくださり,本当に感謝してもしきれません.約二年間,大変お世話になりました.ありがとうございました.\\
\\
\quad 藤井講師には,社会系勉強会にて相談に乗っていただき,研究の進め方について,多岐にわたるご指導,ご助言をいただきました.資料を作成する際にも,研究に取り組む際にも,常に一貫した考えで,目的意識を持って論理的に考えるという姿勢を学ばせていただきました.ご多忙の中,勉強会にてアドバイスをいただき,考察を深めたこ日々は大変貴重で,充実した日々でありました.研究室に入った当初,富士樹海に迷い込んだ子羊だった私が,道を踏み外すことなく,本研究をここまで勧められたのは,藤井先生という北極星が強く光り輝き,導いてくださったからに他なりません.些細なことでも,いつも心よく相談に乗ってくださり学生に親身になってくださる姿勢は私含め多くの学生の心の支えになっていたと思います.私が修士一年の頃,よくゼミでは活発に議論が行われ,今はなきY先生と学生が口論を始めた時にはこの研究室に入って正解だったのかと,いまにも逃げ出したほうがいいのではないかと悪魔の囁きが心の中で誘惑をしていたのですが,そのような緊急事態にもまるで慌てることなく何もなかったことのようにそれを眺める藤井先生の器量の大きい姿が今でも忘れられません.二年間本当にありがとうございました.\\
\\
\quad 内田さんは,自分の研究について一番アドバイスをくださり,実際にコードまで見ていただいた感謝しても仕切れない存在です.研究のけの字も,交通シミュレーションのこの字も,知らない私に一からなんでも優しく教えていただきました.特に私が一週間かかっても解決できなかった問題をほんの一時間ほどで解決した際には,内田さんの後ろからまばゆいばかりの光がさしこみ,菩薩かのような貫禄まで漂っていたことを覚えております.ゼミや社会系勉強会での内田さんの指摘は毎度とても鋭く,ロジックの抜け漏れには誰よりも早く気づき,それを指摘する姿は,研究者でありながらコンサルティングファームのパートナーかのような頭の回転を彷彿とさせておりました.特に私が研究室に入ったばかりの頃,自らの発表資料を先輩の資料をもとに作成した際に,これ資料作成者違う人になってるけど,本当にあなたが作ったんですか?と言われた際には,冷や汗が止まらず,このまま隕石でも落ちて地球がなくなってしまわないかなと感じていたことは,生涯忘れられない思い出となりました.自分の作品には責任を持って,自分で書き上げるという当たり前のことを学ぶことができました.本当にありがとうございました.\\
\\
\quad 阿部さんは,社会系勉強会をはじめとしてCDの発表相手にもなっていただき,自分としては一番距離の近い兄貴のような存在でした.まるで,きびだんごを持たない生まれたての桃太郎のような私に無償でついてきてくださる虎のような存在でした.時にはパワーポイントの使い方を,時には資料のスキャンの方法を,時にはポスターを額縁に入れる方法を教えてくださり,鬼を倒すがごとく研究を進めることができました.勉強会の際には,他の先生方が厳しい質問をしている一方で,阿部さんはコメントと質問を織り交ぜて,学生がしっかり頭を使って答えることができるようにいつも誘導してくださっていて,優しさの溢れる質疑となっていたことが思い出されます.学生が阿部さんの質問を理解できなかった時には,阿部さんが再度「要するに」といって補足してくださるのですが,話している間にどんどん優しさが溢れて行って,最終的にはほとんど答えを言ってしまっていて、全然要せていないことも,阿部さんの人間力と優しさからくるものだなと日々感銘しておりました.私も阿部さんのように知性だけでなく人類を包み込むような優しさと,まるで母親が息子に向けるかのような暖かい笑顔を身につけることのできる人間になりたいと思います.二年間ありがとうございました.\\
\\

%研究室の皆様

%同期

%井上さん
