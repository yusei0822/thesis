\chapter{謝辞}

指導教員であり,本研究の機会を与えていただいた,吉村教授に感謝いたします.研究への姿勢や考え方など,右も左もわからなかった私に一から丁寧に教えていただきました.また,就職活動でもお気遣いいただき,本当に感謝しております.研究を通じて先生に教えていただいた,物事を構造的に捉え,仮説を持って取り組む姿勢は,授業では決して得られない貴重な学びになりました.先生から学んだ仮説思考とも言える考え方は,研究の領域に留まらず,人生においてとても重要なものであると確信しております.今回私は,これまで研究室にはなかった医療というフィールドでの研究を0から始めたわけですが,このようなチャレンジングな機会を与えていただいたこと,また研究内容が社会的意義のあるものだと感じたまま研究に取り組めたことは自分の修士課程にとってとても大きなことでした.本研究での経験や学びを忘れず,社会のために生かして参りたいと存じます.ゼミのディスカッションや研究打ち合わせでは,私に合わせて理解のしやすい説明も織り交ぜながら,医療現場について,研究発表について,教えていただきました.先生とのミーティングは,医療業界全体が抱えるマクロ的な課題をはじめとして,「私の母親の例だと〜」といったようにミクロ的な物事の捉えかた,縦横無尽に課題を探索していく思考の柔軟さに圧倒されることが多く,非常に学びの多い時間でした.とても忙しいはずなのに何冊も本を読み,貪欲に知識を蓄えていく姿勢は,私も今後死ぬまで持ち続けていきたいと心から感じております.これからは,吉村研究室の名に恥じない,社会に価値を提供できる人間になり,吉村研究室の卒業生として世界に名を馳せるよう精進いたします.また,いつも単位のことを気にかけてくださり,本当に感謝してもしきれません.約二年間,大変お世話になりました.ありがとうございました.

%藤井先生

%内田さん

%阿部さん

%研究室の皆様

%同期

%井上さん
