\begin{thebibliography}{99}
 \bibitem{ex_kousei_v1} 厚生労働省, 平成28年版厚生労働白書(2016)
 \bibitem{nursing_management} 田宮菜々子:高齢者にもとづく高齢者施設ケア.33/35(2010)
 \bibitem{SFM} D.HELBING, P.MOLNAR:Social force model for pedestrian dynamics, Physical Review E 41, 4282,(1995)
 \bibitem{ex_pedestrian_simulation_1} 岡崎甚幸:建築空間における歩行のためのシミュレーションモデルの研究その1 磁気モデルの応用による歩行モデル,目本建築学会論文服告集,2S3,111/ll7(1979)
 \bibitem{ex_pedestrian_simulation_2} Kurdi Teknomo、Groria P.Gerilla:Sensitivity Analysis And Validation of a Multi−Agents Pedestrian Model,Journal of the Eastern Asia Society for Transportation Studies,6,198/213(2005)
 \bibitem{ex_personal_space} 渋谷昌三:パーソナル・スペースの形態に閏する考察,山梨医科大学紀要,2,41/49(1985)
 \bibitem{micturition} 安藤正夫:高齢者における排尿障害の実態について,日本泌尿器科学会誌,82,560/564(1991)
 % \bibitem{mates} MATESの引用
\end{thebibliography}
