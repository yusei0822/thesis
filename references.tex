\begin{thebibliography}{99}
 \bibitem{ex_kousei_v1} 厚生労働省, 平成28年版厚生労働白書(2016)

 \bibitem{2025_problem} 厚生労働省,医療と介護を取り巻く現状 (2016)
 %\bibitem{2025_problem}https://www.mhlw.go.jp/file/05-Shingikai-12404000-Hokenkyoku-Iryouka/0000167844.pdf

 \bibitem{population_GDP_relation} 三菱リサーチコンサルティング,日本経済の中期見通し(2019)
 %\bibitem{population_GDP_relation}https://www.murc.jp/wp-content/uploads/2019/04/medium_1904.pdf

 \bibitem{social_security} 内閣府,社会保障給付費の推移等(2016)
 %\bibitem{social_security} https://www5.cao.go.jp/keizai-shimon/kaigi/special/2030tf/281020/shiryou1_2.pdf

 \bibitem{lack_facility_1} みずほ情報総研株式会社,特別養護老人ホームの開設状況に関する調査研究(2016)
 %\bibitem{lack_facility_1} https://www.mhlw.go.jp/file/06-Seisakujouhou-12300000-Roukenkyoku/63_mizuho_1.pdf

 \bibitem{lack_facility_2} 大和総研,超高齢社会における介護問題(2014)
 %\bibitem{lack_facility_2} https://www.dir.co.jp/report/research/policy-analysis/social-securities/20140509_008508.pdf

 \bibitem{turnover_rate} 厚生労働省,平成29年上半期雇用動向調査結果の概況(2018)
%\bibitem{turnover_rate}https://www.mhlw.go.jp/content/12602000/000482541.pdf

 \bibitem{nurse_solution} 厚生労働省,福祉・介護人材確保対策等について(2018)
 %\bibitem{nurse_solution} https://www.mhlw.go.jp/topics/2018/01/dl/tp0115-s01-01-05.pdf

 \bibitem{nursing_management} 田宮菜々子:高齢者にもとづく高齢者施設ケア.33/35(2010)

 %\bibitem{lack_} 厚生労働省,介護老人福祉施設(2017)
 %\bibitem{lack_} https://www.mhlw.go.jp/file/05-Shingikai-12601000-Seisakutoukatsukan-Sanjikanshitsu_Shakaihoshoutantou/0000171814.pdf
 %\bibitem{lack_nurse_1} 厚生労働省,第7期介護保険事業計画に基づく介護人材の必要数について (2018)
 %\bibitem{lack_nurse_1}https://www.mhlw.go.jp/stf/houdou/0000207323.html

 \bibitem{care_robots} 厚生労働省,介護ロボットの開発と普及のための取り組み(2019)
 %\bibitem{care_robots} http://www.techno-aids.or.jp/robot/file01/02shiryo.pdf

 \bibitem{continum_theory} R.L Hughes:A continuum theory for the flow of pedestrians Transportation Research B,Vol 36,No 6,507/535(2002)

 \bibitem{ex_pedestrian_simulation_1} 岡崎甚幸:建築空間における歩行のためのシミュレーションモデルの研究その1 磁気モデルの応用による歩行モデル,目本建築学会論文服告集,2S3,111/ll7(1979)

 \bibitem{ex_pedestrian_simulation_2} Kurdi Teknomo、Groria P.Gerilla:Sensitivity Analysis And Validation of a Multi−Agents Pedestrian Model,Journal of the Eastern Asia Society for Transportation Studies,6,198/213(2005)

 \bibitem{floor_field_model} 柳澤大地,西成活裕:群衆の集団運動と拡張フロアフィールドモデル,応用力学研究所研究集会報告,No.17,ME-S2(2006)

 \bibitem{jiki_model} 岡崎甚幸:建設空間における歩行のためのシミュレーションモデルの研究 その1:磁気モデルの応用による歩行モデル,日本建築学会論文報告集,Vol.283,111/119(1979)

 \bibitem{SFM} D.HELBING, P.MOLNAR:Social force model for pedestrian dynamics, Physical Review E 41, 4282,(1995)

 \bibitem{dimension_pedestrian_model} 山下倫央,副田俊介,大西正輝,依田育士,野田五十樹:一次元歩行者モデルを用いた高速避難シミュレータの開発とその応用,情報処理学会論文誌,vol.53,no.7,1732/1744(2012)

 \bibitem{exopm} 藤井秀樹,西岡智彦,城所直樹,内田英明,吉村忍:拡張1次元歩行者モデルの構築と交差点における歩車混合交通シミュレーション,情報処理学会論文誌,Vol.59,No.3

 \bibitem{no_signal} 三井達郎,矢野伸裕,萩田賢司:無信号横断歩道における高齢者の横断行動と安全対策に関する研究,土木計画学研究・論文集,No.15(1998)

 \bibitem{ped_model} J.Y.S.Lee,W.H.K.Lam,M.L.Tam:Calibration Of Pedestrian Simulation Model for Signalize Crosswalk in Hong Kong.Proceedings of the Eastern Asia Society for Trans Potations Studies,Vol.5,1337/1351(2005)

 \bibitem{interaction} 北川直樹,羽藤英二:疑似最尤法による歩行者と自動車の相互作用モデル,第40回土木計画学会研究発表会(2009)

 \bibitem{NETSIM} A.K.Rathi,A.J.Santiago:Urban Network Simulation:TRAF-NETSIM Program,Transportation Engineering,Vol.116,No.6,734/743(1992)

 \bibitem{AVENUE} 堀口良太,片倉正彦,赤羽弘和,桑原雅夫:都市街路網の交通流シミュレータ-AVENUE-の開発,第13回交通工学研究発表会論文集,33/36(1993)

 \bibitem{mates_1} S.Yoshimura:MATES:Multi-Agent Based Traffic and Environment Simulator - Theory,Implementation and Practical Application,Computer Modeling in Engineering and Sciences,Vol.11,No.1,17/25(2006)

 \bibitem{mates_2} H.Fujii,H.Uchida,S.Yoshimura:Agent-based simulation framework for mixed traffic of cars,pedestrians and trams,Transportation Research Part C: Emerging Technologies,vol.85,234/248(2017)

 \bibitem{mates_3} 藤井秀樹, 仲間豊, 吉村忍:知的マルチエージェント交通流シミュレータMATESの開発第二報:歩行者エージェントの実装と歩車相互作用の理論・実測値との比較,シミュレーション,Vol.25,No.4,274/280(2006)

 \bibitem{mates_4} 吉村忍,藤井秀樹,内田英明,加納達彬:混合交通流シミュレータによる岡山駅前路面電車軌道延伸計画の交通影響評価,交通工学論文集(特集号),Vol.3,No.4,B1/B10(2017)

 \bibitem{dynamic_simulation} J.Bracel’o,J.Casas:Dynamic Network Simulation with AIMSUN,Simulation Approaches in Transportation Analysis,57/98(2002)

 \bibitem{escape_simulation} 吉田孝志,前野義晴,但野紅美子:移動制約者を含む群集の広域避難シミュレーション,第2回SIG-BI(2015)

 \bibitem{necessity} 大脇鉄也,諸田恵二,上原克己:シミュレーションを利用した歩行者自転車混合交通の分離必要度の評価,土木計画学研究・講演集,No.39(CD-ROM)(2009)

 %\bibitem{ex_personal_space} 渋谷昌三:パーソナル・スペースの形態に閏する考察,山梨医科大学紀要,2,41/49(1985)

 \bibitem{micturition} 安藤正夫:高齢者における排尿障害の実態について,日本泌尿器科学会誌,82,560/564(1991)
\end{thebibliography}
