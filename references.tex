\begin{thebibliography}{99}
 \bibitem{ex_kousei_v1} 厚生労働省, 平成28年版厚生労働白書(2016)
 \bibitem{2025_problem} 厚生労働省,医療と介護を取り巻く現状 (2016)
 %\bibitem{2025_problem}https://www.mhlw.go.jp/file/05-Shingikai-12404000-Hokenkyoku-Iryouka/0000167844.pdf
 \bibitem{population_GDP_relation} 三菱リサーチコンサルティング,日本経済の中期見通し(2019)
 %\bibitem{population_GDP_relation}https://www.murc.jp/wp-content/uploads/2019/04/medium_1904.pdf
 \bibitem{social_security} 内閣府,社会保障給付費の推移等(2016)
 %\bibitem{social_security} https://www5.cao.go.jp/keizai-shimon/kaigi/special/2030tf/281020/shiryou1_2.pdf
 \bibitem{lack_facility_1} みずほ情報総研株式会社,特別養護老人ホームの開設状況に関する調査研究(2016)
 %\bibitem{lack_facility_1} https://www.mhlw.go.jp/file/06-Seisakujouhou-12300000-Roukenkyoku/63_mizuho_1.pdf
 \bibitem{lack_facility_2} 大和総研,超高齢社会における介護問題(2014)
 %\bibitem{lack_facility_2} https://www.dir.co.jp/report/research/policy-analysis/social-securities/20140509_008508.pdf
 \bibitem{turnover_rate} 厚生労働省,平成29年上半期雇用動向調査結果の概況(2018)
 %\bibitem{turnover_rate}https://www.mhlw.go.jp/content/12602000/000482541.pdf
 \bibitem{nurse_solution} 厚生労働省,福祉・介護人材確保対策等について(2018)
 %\bibitem{nurse_solution} https://www.mhlw.go.jp/topics/2018/01/dl/tp0115-s01-01-05.pdf
 \bibitem{nursing_management} 田宮菜々子:高齢者にもとづく高齢者施設ケア.33/35(2010)
 %\bibitem{lack_} 厚生労働省,介護老人福祉施設(2017)
 %\bibitem{lack_} https://www.mhlw.go.jp/file/05-Shingikai-12601000-Seisakutoukatsukan-Sanjikanshitsu_Shakaihoshoutantou/0000171814.pdf
 %\bibitem{lack_nurse_1} 厚生労働省,第7期介護保険事業計画に基づく介護人材の必要数について (2018)
 %\bibitem{lack_nurse_1}https://www.mhlw.go.jp/stf/houdou/0000207323.html
 \bibitem{care_robots} 厚生労働省,介護ロボットの開発と普及のための取り組み(2019)
 %\bibitem{care_robots} http://www.techno-aids.or.jp/robot/file01/02shiryo.pdf
 \bibitem{SFM} D.HELBING, P.MOLNAR:Social force model for pedestrian dynamics, Physical Review E 41, 4282,(1995)
 \bibitem{ex_pedestrian_simulation_1} 岡崎甚幸:建築空間における歩行のためのシミュレーションモデルの研究その1 磁気モデルの応用による歩行モデル,目本建築学会論文服告集,2S3,111/ll7(1979)
 \bibitem{ex_pedestrian_simulation_2} Kurdi Teknomo、Groria P.Gerilla:Sensitivity Analysis And Validation of a Multi−Agents Pedestrian Model,Journal of the Eastern Asia Society for Transportation Studies,6,198/213(2005)
 % \bibitem{ex_personal_space} 渋谷昌三:パーソナル・スペースの形態に閏する考察,山梨医科大学紀要,2,41/49(1985)
 \bibitem{micturition} 安藤正夫:高齢者における排尿障害の実態について,日本泌尿器科学会誌,82,560/564(1991)
\end{thebibliography}
