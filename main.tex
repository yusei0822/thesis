%%
%%
\documentclass[12pt,a4j,twoside,openany,titlepage,dvipdfmx]{jsbook}
%
%% パッケージ
\usepackage{amsmath}  % アメリカ数学会の数式拡張
\usepackage{amssymb}  % アメリカ数学会の数学記号拡張
\usepackage[dvipdfmx]{graphicx} % 画像コマンド拡張
\usepackage[margin=30truemm]{geometry} % 余白調整
\usepackage[nottoc]{tocbibind} % 目次に図目次・表目次を表示
\usepackage {diagbox} % 表の斜線のやつ
\usepackage{bm} % ベクトル表記
%
%% subsubsection まで目次に表示
\setcounter{secnumdepth}{3}
\setcounter{tocdepth}{3}
%
%% 行間を調整 (30 行になるように調整)
\linespread{1.18}
%
%% 表紙
\renewcommand{\maketitle}{
\begin{titlepage}
 \setcounter{page}{0}
 \begin{center}
  {\Large 東京大学大学院工学系研究科システム創成学専攻} \\[8ex]
  {\Large 修士論文} \\[12ex]
  {\huge 高齢者支援施設における \\ マルチエージェントシミュレーション } \\
  {\LARGE Multi-agent simulation in nursing homes} \\
  \vfill
  {\Large \today ビルド} \\[12ex]
  {\Large 指導教員 吉村忍 教授} \\[12ex]
  {\Large 学籍番号 37-186421} \\[8ex]
  {\Huge 紫安勇成}
 \end{center}
\end{titlepage}
}
%
%%%%%%%%%%%%%%%%%%%%%%%%%%%%%%%%%%%%%%%%
%
\begin{document}
%
%% 前付け
\frontmatter
\maketitle % 表紙
\tableofcontents % 目次
\listoffigures % 図目次
\listoftables % 表目次
%
%% 本文
\mainmatter
\thispagestyle{myheadings}
\markboth{_}{_}
\mbox{}\newpage

\chapter{序論}

\section{研究背景}

高齢者施設は,現代社会の基盤となるシステムである.一方で,少子高齢化やそれに伴う介護士の不足は,医療サービス利用者にとって大きな問題である.これらの問題を解決するために,介護士の労働環境の改善や医療施設の拡充等が検討されている.医療現場は,一旦変更してしまうと容易に元に戻すことが難しい.しかも医療現場は非常に複雑であり,サービスの被提供者のプライベートや安全と密接に関連していることから,実験を行うこと自体が時間・コスト・安全の面から現実的ではない.このため,最新技術の導入等の実験を行い,それらの効果検証が出来る医療シミュレータの開発が急を要しているものの,医療現場はプライベートな空間であり,これまで現実データを十分に獲得することが出来ず,有用なシミュレータの構築が難しかった.しかし,今後の日本における医療の重要性を考えると,個人の特性や意志を持った主体として介護者,被介護者を取り扱い,それらの詳細な相互作用を取り入れたシミュレータの構築が必要であると考えられる.

\subsection{日本の抱える諸課題}

我が国では,世界に先駆けて少子高齢化が深刻化している.1950年時点で5%に満たなかった高齢化率(65歳以上人口割合)は,1985年には10.3%,2005年には20.2%と急速に上昇し,2018年は27.5%と過去最高となっている.2060年まで一貫して高齢化率は上昇していくことが見込まれており,2060年時点では約2.5人に1人が65歳以上の高齢者となる見込みである\cite{ex_kousei_v1}.
現状でも日本の高齢化率は類見ないほどであり,2位のイタリア22.7% を5%近く引き離している.
直近では,2025年問題が迫っている.2025年問題とは,1947~1949年生まれのいわゆる団塊の世代が65歳以上になり,高齢化率がさらに上昇するという問題のことである.\cite{2025_problem}
このような少子高齢化に伴い,様々な問題が引き起こされている.生産年齢人口の減少により,生産活動と消費活動の衰えが経済全体の停滞をもたらすことはもちろん,社会保障費の増大による国家財政の不安定化も免れない.
「成長会計」の考えに従えば,マクロの経済成長率は,労働投入量の伸び,資本投入の伸び,それ以外の要素の伸びに要因分解される.このように,労働力供給は一国の潜在成長力を規定する要因の一つであるため,労働力人口の減少が今後長期的に経済成長を抑制するとの見方が共通認識となっている.\cite{population_GDP_relation}
また,社会保障費の増大も深刻であり,2025年度には対GDP比で24.4%になるとされている.\cite{social_security}

\begin{figure}[htb]
 \begin{center}
 \includegraphics[scale=0.6]{figures/population_GDP_relation}
 \caption[経済衰退]{経済衰退 \label{population_GDP_relation}}
 \end{center}
\end{figure}

\subsection{医療現場のマクロ課題}

介護の現場には高齢化による高齢者の増加が,逼迫した問題としてのしかかっている.要介護の高齢者が増えるにつれて,介護や医療のサービスを提供する施設は増加している.図\ref{nurse_facility_increase}のように,有料老人ホームは10年の間で施設数が10倍以上に増えている.

\begin{figure}[htb]
 \begin{center}
 \includegraphics[scale=0.4]{figures/nurse_facility_increase}
 \caption[高齢者向け住まい・施設の定員数]{高齢者向け住まい・施設の定員数 \label{nurse_facility_increase}}
 \end{center}
\end{figure}

しかし,施設数は増加しているものの,十分な供給量には達していない.「特別養護老人ホーム」「介護老人保健施設」「介護療養型医療施設」の3類型の定員数(介護療養型医療施設は病床数)合計にしめる65歳以上人口割合は軒並み4%を切っており,十分な施設供給ができているとは言えない.\cite{lack_facility_1}
特に特別養護老人ホームの不足は深刻だ.2014年の厚生労働省の発表では,特別養護老人ホームの入所申込者数(いわゆる待機老人数)が2013年度に52.4万人と発表するなど,介護施設の不足が改めて確認された.特別養護老人ホームへの入所申込者数は,2009年度調査の42.1万人から約10万人増加している.特別養護老人ホームの数は2012年度時点でで全国で7,605施設,総定員数は50.7万人である.つまり2012年時点ですら,総定員数以上の入所希望者がおり,施設が不足していたことがわかる.\cite{lack_facility_2}

また,施設数の増加によって必要となる介護士の数は十分とは言えない.
2018年に厚生労働省は,全国の市町村が策定した第7期の介護保険事業計画を踏まえた,必要介護者数を公表した.
それによれば,介護職員の需要は2020年度で216万494人,2025年度で244万6562人に増える.足下の2016年度の実績は189万8760人で,2025年度との差は54万7802人であり,毎年6万人の補充をし続けなければならない.

しかし,目先の施設数や介護士数を増やしたところで,他にも大きな問題点が残留している.介護士の離職率の高さである.
常勤労働者の離職率を比較して見ると,産業全体の離職率11.6%であるのに対して介護職員の離職率は19.0%となっている.
この状況は,直近7年間で変動していない.平成22年度の産業全体の離職率は14.5%であるのに対し,介護職員の離職率は17.8%であって,産業全体の離職率を介護職員の離職率が下回ったことがない.\cite{turnover_rate}
介護労働安定センターの平成25年度介護労働実態調査によれば,「仕事内容の割に賃金が低い」と「人手が足りない」の二項目が介護士の不満であり,人手不足が人手不足を生むという介護士の離職率を高める負の連鎖が生まれている.

前述の通り, 厚生労働省は毎年6万人の介護職員増員を目標としている.増員施策として大きく「参入促進」「労働環境・処遇の改善」を行っている.\cite{nurse_solution}
参入促進に関して,地域住民や学校の生徒に対する介護や介護の仕事の理解促進や若者・女性・高齢者など多様な世代を対象とした介護の職場体験といった施策は従来から行われていた.
それに加えて,三つの新たな施策が行われている.
まず,介護に関する入門的研修の実施からマッチングまでの一体的支援事業の創設である.介護未経験者の介護分野への参入のきっかけを作るとともに,参入障壁となっている介護に関する様々な不安を払拭するため,介護業務の入門的な知識・技術の修得のための研修を導入し,介護人材のすそ野を拡げ,中高年齢者など多様な人材の参入を促進するという施策である.
次に,将来の介護サービスを支える若年世代の参入促進事業である.介護に関する教育機関である介護福祉士養成施設において,将来の介護現場を担う世代に対する介護の専門性や意義を伝達する取組や,今後増加することが予想される留学生への日本語学習支援等による質の高い人材の養成・確保に係る取組を推進している.
加えて,介護福祉士国家資格の取得を目指す外国人留学生の受入環境整備事業の創設も行っている.介護施設等による外国人留学生への奨学金等の支給と介護福祉士資格の取得を目指す留学生と受入介護施設等とのマッチング支援によって,外国人の介護士の活用を始めた.
また,労働環境・処遇の改善について,管理者へ雇用改善方策の普及を主に行っている.具体的には,管理者に対する雇用管理改善のための労働関係法規,休暇・休職制度等の理解のための説明会の開催などが行われている.

こうした施策によって介護士の母数そのものを増やすことも重要であるが,介護士1人当たりが対応できる要介護者を増やすことも同程度に必要なことである.
介護士の労働生産性は他業種に比べて低いとされている.介護職員(介護保険施設)の1人当たり付加価値労働生産性は430万円(2010年)と試算される.これは製造業の760 万円に対して56.6%,非製造業の641万円に対して67.1%と,およそ4から5割低い水準にある.
以上のことからも,一人当たりの介護量を増やす可能性はあると考えられる.
政府も現状,この課題に着手しており,介護人材キャリアアップ研修支援や地域包括ケアシステム構築に資する人材育成, 認知症ケアに携わる人材育成のための研修などを行っている.
さらに生産性向上の実現可能性を高めるものとして,医療技術の発達は目覚ましい.

\subsection{高齢者施設運営における課題}
我が国では,平成12年に医療保険制度に加えて,「介護保険制度」が導入された.介護保険の総費用は,平成12年から平成21年にかけて,3.6兆円から7.4兆円へ年10%を超える伸びで増大し,月間利用者も184万人(平成12年度)から344万人(平成19年度)と増加した.当初218万人だった要介護認定者が,現在数倍になっており,要介護高齢者の状態像で,認知症・廃用症候群・脳卒中の3大症状の対策が重要であると言われている.
今村ら\cite{nursing_management}によると,今後の施設運営で考慮すべき問題点は大きく8点挙げられている.

\begin{itemize}
 \item 「昭和ヒトケタ世代」から「第一次ベビーブーマー」へ移行した際の問題
 \item 認知症の増加の問題
 \item 単独世帯の増加の問題
 \item 都市部の高齢化の問題
 \item 看取り(死亡)の場所と体制の問題
 \item 介護のリハビリテーションの問題
 \item 医療と介護の連携の問題
 \item 介護従事者の確保の問題
\end{itemize}

以上のように,今後高齢者施設を適切に運営していくためには,すぐれた療養環境の提供と個別ケアが求めらていると言える.また,施設だけで完結するのではなく,地域や家族の協力のもと,適切な在宅ケアを提供していく必要がある.

\subsection{国の取り組み}

前述の通り, 厚生労働省は毎年6万人の介護職員増員を目標としている.増員施策として大きく「参入促進」「労働環境・処遇の改善」を行っている.
参入促進に関して,地域住民や学校の生徒に対する介護や介護の仕事の理解促進や若者・女性・高齢者など多様な世代を対象とした介護の職場体験といった施策は従来から行われていた.
それに加えて,三つの新たな施策が行われている.
まず,介護に関する入門的研修の実施からマッチングまでの一体的支援事業の創設である.介護未経験者の介護分野への参入のきっかけを作るとともに,参入障壁となっている介護に関する様々な不安を払拭するため,介護業務の入門的な知識・技術の修得のための研修を導入し,介護人材のすそ野を拡げ,中高年齢者など多様な人材の参入を促進するという施策である.
次に,将来の介護サービスを支える若年世代の参入促進事業である.介護に関する教育機関である介護福祉士養成施設において,将来の介護現場を担う世代に対する介護の専門性や意義を伝達する取組や,今後増加することが予想される留学生への日本語学習支援等による質の高い人材の養成・確保に係る取組を推進している.
加えて,介護福祉士国家資格の取得を目指す外国人留学生の受入環境整備事業の創設も行っている.介護施設等による外国人留学生への奨学金等の支給と介護福祉士資格の取得を目指す留学生と受入介護施設等とのマッチング支援によって,外国人の介護士の活用を始めた.
また,労働環境・処遇の改善について,管理者へ雇用改善方策の普及を主に行っている.具体的には,管理者に対する雇用管理改善のための労働関係法規,休暇・休職制度等の理解のための説明会の開催などが行われている.
こうした施策によって介護士の母数そのものを増やすことも重要であるが,介護士1人当たりが対応できる要介護者を増やすことも同程度に必要なことである.
その実現可能性を高めるものとして,医療技術の発達は目覚ましい.

\subsection{医療技術の発達}

医療技術の一例として,医療ロボットとハイテク福祉機器について言及したい. 厚生労働省は,2040年を展望した中長期ビジョンである「未来イノベーション WG 」の取りまとめを踏まえた医療福祉分野における取組を検討し,2019年度中に具体化することを目標としており,医療・介護現場のハイテク化に努めている\cite{care_robots}.
第一に,介護ロボットとは,情報を感知し,判断し,動作するという3要素を有する知能化した機械システムのことである.具体的には,図\ref{care_robots}のように,要介護者の移乗支援を行う装着型パワーアシストや,要介護者の排泄支援を行う自動排泄処理装置などが該当する.政府としても,介護ロボットを現場に導入することを支援しており,地域医療介護総合確保基金を始めとした各種助成金の設立やニーズ・シーズ連携協調のための協議会の設置を行っている.

\begin{figure}[htb]
 \begin{center}
 \includegraphics[scale=0.4]{figures/care_robots}
 \caption[介護ロボットの例]{介護ロボットの例 \label{care_robots}}
 \end{center}
\end{figure}

第二に,ハイテク福祉機器の事例を説明する.ハイテク福祉機器とは新しい要素(アミューズメントやアート等)を取り入れたり,ICTやIoTといった技術を採用した福祉機器のことである.ハイテク福祉機器の登場は,停滞気味であった福祉機器市場全体を押し上げる要因となるほどのインパクトを持っており,H.C.R.2018という国際フォーラムも開かれるほど注目を集めている.図\ref{yubicommnnication}のように,具体的な製品として,有限会社オフィス結シェアの提供する指伝話コミュニケーションパックは,指伝話メモリで作成したコンテンツ集で,指伝話メモリのカードを選択してさまざまな機能を実現することができる.指伝話プラスや指伝話文字盤など他のアプリをホーム画面に戻らずに呼出,SMSやメールの送信,ウェブサイトやYouTubeを開く,iOSのショートカットを用いてiOSの機能を利用するといったことを行うための実用的なサンプルセット集である.会話が困難であったり,日常生活での移動が困難な要介護者への支援として各施設に導入され,介護士の負担軽減の一助となっている.

\begin{figure}[htb]
 \begin{center}
 \includegraphics[scale=0.4]{figures/yubicommunication.png}
 \caption[ハイテク福祉機器の例]{ハイテク福祉機器の例 \label{yubicommnnication}}
 \end{center}
\end{figure}

また,上記のような医療に関する技術は,高齢者自身の自立生活支援,高齢者介護の支援,生活の質(QOL)と快適性の高揚とその維持など,高齢者に対する生活支援分野においては,その使用者が被介護者か介護者により自立支援技術と介護支援技術に分けられる.\\
・自立支援技術 \\
排泄,入浴,調理,食事,就寝・起床,洗濯,清掃,義肢・装具,移乗,生活圏移動 \\
・介護支援技術 \\
排泄,入浴,清拭,褥瘡予防,食事,就寝・起床,移載・移動,監視 \\
また,これら技術には開発の優先順位が設けられている.以下の表\ref{tech_classification}では,高齢者を自立意識が高い,低い,認知症の3パターンで分類したものと,被介護者の状態として全介護,半介護,身体異常の3パターンで分類したものとのマトリクスを示している.

\begin{table}[htb]
  \caption[介護技術の分類]{介護技術の分類}
  \label{tech_classification}
  \centering
  \begin{tabular}{r|c|c|c}
     & 全介護 & 半介護 & 身体異常 \\ \hline
    自立意識強い & 自立支援 & \quad & \quad \\
    自立意識弱い & \quad & \quad &  \quad \\
    認知症 & \quad & 介護支援 & \quad \\
    \end{tabular}
\end{table}

・自立支援 \\
自立意欲が高いにもかかわらず自立できていないことに対する支援機器が必要 \\
・介護支援 \\
一部介護を要する自立意欲の弱い高齢者には肉体的・精神的・時間的に大きな負担 \\
表\ref{tech_classification}にあるように,全介護の状態でありながらも,自立意識の高い被介護者と,半介護の状態でありながら,自立意識が弱かったり,認知症である場合に,それぞれ自立支援機器と介護支援機器の開発が必要とされる.

\subsection{医療技術導入における課題}

前述の支援機器の実用化には以下のように多くの問題点が存在する.

\begin{itemize}
 \item 性能不十分
 \item 操作複雑
 \item 寸法・重量
 \item 高価
 \item 危険
 \item 公害
 \item プライバシーの侵害
\end{itemize}

これら問題により,新技術の導入は医療サービス被提供者にとってはもちろん,医療機関にとって簡単に行うことができない.これら技術の導入が進んで来なかった背景として,そもそも技術として不完全であることに加え,現場の忙しい看護師や医師たちが簡単に使えるようなものでなければいけないことなど多くの制約があり,中でも,導入した際の効果検証ができないことが,意思決定の大きなボトルネックとなっている.また,病院は収益が順調に出ている状態だと,リスクをとって環境を改善するインセンティブが湧きづらく,こういった技術導入へのインセンティブが働かないことも大きな課題として挙げられる.

以上のように,最新技術の導入等の実験を行い,それらの効果検証が出来る医療シミュレータの開発が急を要しているものの,医療現場はプライベートな空間であり,これまで現実データを十分に獲得することが出来ず,有用なシミュレータの構築が難しかった.しかし,今後の日本における医療の重要性を考えると,個人の特性や意志を持った主体として介護者,被介護者を取り扱い,それらの詳細な相互作用を取り入れたシミュレータの構築が必要であると考えられる.

シミュレーションを行う際に気を付けなければならないのは,シミュレーションで用いる行動ルールのパラメータの妥当性とその客観性である.実際の介護の動きの計測結果から客観的に抽出されるルールやパラメータを入力とするシミュレーションが理想的である.しかし現在のところ,行動ルールのパラメータを抽出することを可能にするほどの精度の高い人流計測をするための研究はあまりされていない.これは一つには現在主流である単純な画像解析による手法の限界,もう一つには介護という環境がプライベートな空間であり,そもそも計測をできるような環境にない,という事が挙げられる.

\section{目的}

我が国日本では,医療技術の研究が盛んに行われているのにも関わらず,それらの導入・浸透には至っていない.そこで本論文では,各医療機関がそれら技術の導入の意思決定につながるシミュレーションモデルを構築することを目的とする.本研究の第一ステップとして,現在医療の現場で大きな課題となっている排泄介助にスコープを当て,排泄介助のシミュレーション上で技術の性能評価を行い,それによって技術の導入促進の意思決定に資するプラットフォームの構築をおこなう.

\section{本論文の構成}

1章では本論文の研究背景として,日本,医療界の諸課題について説明を行った.それに加えて課題の解決を目指す技術の紹介を行い,それらを導入する上での課題点を整理することで本論文の目的を示した.
2章では,提案手法についての説明をしている.3章では提案手法の数値実験により手法の検証を行う.4章では3章で得られた結果をまとめ,それらから得られる示唆についての考察を行い論文のまとめとする.

\thispagestyle{myheadings}
\markboth{_}{_}
\mbox{}\newpage

%2章
\chapter{提案手法}

本研究で対象とするエージェントシミュレーションの大きな特徴は,介護の対象となる高齢者の運動機能や認知機能の低下に大きなバリエーションがあると同時にも,介護者側にも国家資格をもった介護福祉士から,介護ヘルパー,ボランチィアスタッフまで技能や知識,経験に大きなバリエーションがあることである.そうしたことを念頭に置いた上で,本研究では介護者エージェント,被介護エージェント,環境としての高齢者施設の基本モデリングを検討した.図\ref{concept_simulation}に概念図を示す.黒で示される介護者が、自身が持つ視野の中で水色で示される被介護者を観測する.

\begin{figure}[htb]
\begin{center}
\includegraphics[scale=0.6]{figures/concept_simulation.png}
\caption[シミュレーションの概念図]{シミュレーションの概念図 \label{concept_simulation}}
\end{center}
\end{figure}

\section{知的マルチエージェントモデル}

介護行動は社会系の複雑現象である.私たちが行動を起こす際に,認知症による自己の生理機能への理解が周囲に与える影響を懸念することはあっても,それの繰り返しによって大きな事故につながると理解している人は少ない.しかし,個人レベルでは,手すりに捕まる,他の歩行者に接触しないようにするといった比較的単純なルールに従い行動しているが,それらの個人行動が多種・多量に存在し,相互作用することによって全体としては非常に複雑な現象となる.複雑系を解析する手法の一つとして,マルチエージェント手法がある.しばしば,セルオートマトンが複雑系のシミュレーションに用いられ,セルオートマトンに基づくシミュレーションの研究事例もいくつか存在する.これに対して,本シミュレータでは,人間という知的レベルの高い主体が多数集まり相互作用を起こす介護現象をより精緻に再現するために,情報を知覚し,それを基に自律的に行動を起こす主体を知的エージェント,それを取り巻く世界を環境と定義し,シミュレーションの構造はマルチエージェントのフレームワークに基づき構築している.そこで,これを知的マルチエージェントモデルと呼ぶ.

\subsection{知的エージェントの構築}
図\ref{intelligent_agent}に知的工一ジェントのイメージを示す.知的エージェントは,情報を知覚するセンサーと動作を実行する作用器を持っている.また,エージェント自身の思考プロセスを保持しており,センサーから得られた情報と自分の有する知識と判断基準に基づき自律的に行動を決定し,作用器を通して行動を起こし,環境に働きかける.センサー,作用器,思考は知的エージェントが実際に適用される時点で,問題に応じて定義される.図\ref{agent_modeling}にエージェントと環境の相互作用の様子を模式的に示す.介護者エージェントが自らの行動により環境に影響を与え、その環境によって被介護者エージェントが影響を受けることになる.ある主体の動きによって系全体の動きが規程され,複雑な現象が創発する.

\begin{figure}[htb]
\begin{center}
 \includegraphics[scale=0.6]{figures/intelligent_agent.png}
 \caption[知的エージェントの模式図]{知的エージェントの模式図 \label{intelligent_agent}}
\end{center}
\end{figure}

\begin{figure}[htb]
\begin{center}
 \includegraphics[scale=0.6]{figures/agent_modeling.png}
 \caption[本シミュレーションにおける環境とエージェント]{本シミュレーションにおける環境とエージェント \label{agent_modeling}}
\end{center}
\end{figure}

\subsection{Social force model}

本研究では,高齢者施設内で介護者が高齢者のトイレ介護のために空間移動するプロセスをモデリングするために,Social force model(SFM)\cite{SFM}という歩行者モデリング理論を用いる.Social Force Modelは,歩行者を2次元の粒子であると仮定し,その粒子に以下の4つの力が働くと仮定するモデルである.

\begin{itemize}
 \item 移動目標に近づく力
 \item 他のエージェントからの斥力
 \item 壁などの環境からの斥力
 \item 魅力的な環境への引力
\end{itemize}

移動目標に近づく引力は,エージェントが当初想定していたコースからはずれてしまった場合に目的地の進行方向へと曲げるように働く力のことであり,他のエージェントや壁などからの斥力は,エージェント間,あるいは壁とエージェント間との距離や,お互いの進行方向から決定される反発的な力のことである.魅力的な環境への引力では,友人やショーウィンドー,高齢者支援施設の中では手すりのような,歩行者にとって近づくことのインセンティブが発生するようなものへの引力のことである.これら4つの力は以下のように数式で表現される.

\begin{equation}
\begin{split}
 F_α(t)=F_α^o(υ_α,υ_α^0e_α)&+\sum_{β}F_αβ(e_α,r_α-r_β)\\
 &+\sum_{B}F_αβ(e_α,r_α-r_B^α)\\
 &+\sum_{i}F_αi(e_α,r_α-r_i,t)
\end{split}
\end{equation}

右辺第一項が移動目標に近づく力,第二項が他のエージェントからの斥力,第三項が壁などの環境からの斥力,第四項が魅力的な環境への引力を表している.各エージェントはタイムステップごとに以上の力を計算して,あらかじめ設定された最大歩行速度を超えないように歩行速度を更新する.

\subsection{シミュレーションフロー}

本研究におけるシミュレーションの流れの概略図を図\ref{simulation_flow}に示す。まず、高齢者施設、介護者、被介護者といった空間を構成する要素を環境として構成し、その後実際にシミュレーションを開始する。タイムステップごとに、各エージェントの内部状態を変化させ、介護シミュレーションを行っていく。被介護者の場合は、時間経過で尿量を加算し、エージェントごとに設定されている閾値を超えた時点で介護アラートを出す。介護者は、自分の周りで介護アラートが出たタイミングで、自らと最も距離の近い被介護者のもとへの介護に向かう。これを繰り返すことがシミュレーションが進んで行く。

\begin{figure}[htb]
\begin{center}
 \includegraphics[scale=0.6]{figures/simulation_flow.png}
 \caption[シミュレーションフロー]{シミュレーションフロー \label{simulation_flow}}
\end{center}
\end{figure}

\subsection{介護ペア選択アルゴリズム}

上述のように、被介護者がアラートを発した時にどの介護士とマッチングさせるのかというのがシミュレーション上必要になる。本研究では、各タイムステップごとにある被介護者がアラートを出した時点で、その被介護者と介護可能な介護士との距離を計算し、ペアになりうる介護士と被介護者のペア候補配列を作成していき、その中で全探索を行うことで、最も距離の近いペアを作成して行くこととする。

\section{仮想環境}
本シミュレータにおける環境とは,高齢者施設とそこに存在する介護者、被介護者構造を指す.高齢者施設のモデル化はそれ自体が交通流シミュレータの汎用性・拡張性を実現する上で重要な課題である.本シミュレータでは,介護者・被介護者は基本的に.自由移動を行うことができる二次元平面を想定している。

\subsection{高齢者施設}

今回の開発では,高齢者施設を表現するための第1ステップとして,Social Force Modelを利用するため自由に歩行できる連続空間を対象とし、壁に囲まれた移動可能な二次元平面を作成した。

\subsection{介護者エージェント}

介護者工一ジェントは2次元の高齢者施設に存存するため,方向と速度の制御を行わなければならない.今回は,歩行者の行動モデルとしてSocial Force Modelを採用した.歩行者シミュレーションの研究分野においては、様々なモデル化が検剤されている\cite{ex_pedestrian_simulation_1,ex_pedestrian_simulation_2}.たとえば,磁気モデルを用いると多方向に歩行するエージェントの相互作用を効率的に記述することが可能である.しかし,今回の研究のように閉二次元平面での歩行を対象とする場合には,部屋の中を目的地に対してどう動いて行くかを単純に記述するモデルで十分である.また歩行者モデルとしてあまり複雑なモデルを採用すると計算量が増大しシミュレータの大規模化が困難であることから,本研究ではSocial Force Modelを採用した.以下で歩行者工一ジェントの特徴と挙動について述べる.

\subsection{被介護者エージェント}

被介護者エージェントについては、介護者エージェントと同様にSocial Force Modelを軸に、歩行者エージェントを作成し、それに加えてエージェントの状態によって時系列的に発生する要介護行動を実装した.高齢者の排尿に関する実態研究 \cite{micturition} によると、排尿障害症状を自覚している人は男子が38%、女子が23%と高い水準にあり、男子では排尿困難症状が多く、女子では頻尿を訴える例が多かった。また明らかな尿失禁を抱えているのにも関わらず、その存在を知られたくないという心理が半数以上の人に認められたことも挙げられている。これらから、実際にトイレに行きたいと思っているかどうかの認知についてと、トイレで正常に排尿を行えるのかどうかといった機能について、被介護者のバリエーションを設けることとした.

\thispagestyle{myheadings}
\markboth{_}{_}
\mbox{}\newpage

% 3章
\chapter{数値実験}

\section{実験条件}

% 何メートル四方の環境
% 介護者の数、被介護者の数
% 2時間の中で排泄介助を行う
% 初期値として尿量を与える

\section{介護挙動の基本的な検証}

% 人が1日に何回トイレに行くのかデータと実験値を比較させることによって,ある程度表現できていることを示したい.

\section{評価指標}

% 合計我慢時間と成功回数で評価する.
% 十分な量に達していないときに介護を行なった場合は過剰介護,待たせてしまった場合はお漏らし処理などの追加業務が発生する.

\section{結果および考察}

\thispagestyle{myheadings}
\markboth{_}{_}
\mbox{}\newpage

% 4章
\chapter{モデルの検証}

本章では,高齢者施設を模した介護空間が環境として設定できているか,使用した歩行者モデルのSocial Force Modelが正しく動いているか,作成したシミュレータの中で介護シミュレーションが正しく機能しているかについて検証する.Social Force Modelの検証では,歩行者しかいない環境における基本的な可視化をおこない,目的への斥力や他エージェントからの斥力が働いているかを確認する.介護シミュレーションの検証では,被介護者がアラートを出した時に介護者が経路選択を行い,あらかじめ設定した介護行動が行われるかを可視化する.また、複数の介護アラートが発生した時に,あらかじめ設定したロジックに基づき介護対象を選択できるかについても検証を行う.

\section{環境設定の検証}

環境設定の検証は,いくつかの二次元平面を実際に作成し,可視化することで行う.また,その中で介護者・被介護者も同様に可視化する.被介護者は通常状態とアラートを出している状態で色を変える.まずは図\ref{environment_v1}にあるように,簡単な二次元平面と歩行者を可視化した.次に,図\ref{environment_v2}のように,同じ二次元空間の中で,介護者と被介護者を可視化し,被介護者の状態によって色が変わった状態も可視化できた.同様に車椅子に乗っている被介護者を想定した被介護者エージェントや,技術によるサポートを受けている被介護者を想定した被介護者エージェントも同様に,色や形を変えることで表現することができる.本シミュレータでは,介護空間を記述するための必要条件である,二次元平面と介護者,被介護者を再現することができると言える.

\begin{figure}[htb]
\begin{center}
 \includegraphics[scale=0.4]{figures/environment_v1}
 \caption[二次元空間の可視化]{二次元空間の可視化 \label{environment_v1}}
\end{center}
\end{figure}

\begin{figure}[htb]
\begin{center}
 \includegraphics[scale=0.4]{figures/environment_v2}
 \caption[介護者・被介護者の可視化]{介護者・被介護者の可視化 \label{environment_v2}}
\end{center}
\end{figure}

\section{Social Force Modelの検証}

Social Force Modelの検証は,環境設定の検証で作成した環境とは異なる二次元空間を作成し,その中でエージェント同士の相互作用が行われているかを確認する.歩行者エージェント作成し,手入力で目的地の配列を初期条件として与え,一定時間ごとに同じスタート地点にエージェントを発生させ,同じ経路を設定する.現在地から目的地までは,Social Force Modelで定義されている目的地に向かう力と他のエージェントからの斥力,壁からの斥力,魅力的な場所からの引力(本検証では用いない)の合力が働くので,例えば同じ経路上ですれ違う場合は,たとえ経路が同じであったとしても互いが互いを避けて通るようになるはずである.

まずは,環境として図\ref{SFM_environment}を準備する.二次元空間の中に壁を作成し,1の位置から歩行者エージェントが一定時間ごとに発生するようにする.歩行者エージェントは,1から順番に1,2,3,4,3,2,5,2,1,6という経路の目的地群を与え,現在の目的地であるサブゴールに到着すると,次の目的に向かって移動していく.

\begin{figure}[htb]
\begin{center}
 \includegraphics[scale=0.6]{figures/SFM_environment}
 \caption[SFMの検証における環境設定]{SFMの検証における環境設定 \label{SFM_environment}}
\end{center}
\end{figure}

その様子を図\ref{SFM_visual}に可視化した.シミュレーションでは,目的地に到着後,壁にぶつからないように折り返し,他のエージェントとすれ違う際に,目的地間の最短経路を通らず,お互いがお互いを避ける挙動を確認することができた.本シミュレータでは,介護空間を記述するための必要条件である,経路選択と目的地までの移動を再現することができると言える.

\begin{figure}[htb]
\begin{center}
 \includegraphics[scale=0.6]{figures/SFM_visual}
 \caption[SFMの検証における可視化結果]{SFMの検証における可視化結果 \label{SFM_visual}}
\end{center}
\end{figure}

\section{介護シミュレーションの検証}

本章では最後に,介護シミュレーションについての検証を行う.本シミュレーションでは,被介護者のバリエーションとして以下の3種類を実装している.

\begin{itemize}
 \item 適切なタイミングで介護アラートを出すエージェント
 \item 早いタイミングで介護アラートを出すエージェント
 \item 遅いタイミングで介護アラートを出すエージェント
\end{itemize}

本研究では、排泄介助を対象としているので、適切なタイミングで介護アラートを出すエージェントを健常者(技術のサポートを受けている被介護者),早いタイミングで介護アラートを出すエージェントを頻尿の被介護者,遅いタイミングで介護アラートを出すエージェントを認知症の被介護者と呼ぶことにする.それぞれを可視化した図が,図\ref{elderly_v1},図\ref{elderly_v2},図\ref{elderly_v3}である.

\begin{figure}[htb]
\begin{center}
 \includegraphics[scale=0.5]{figures/elderly_v1.png}
 \caption[健常者(介護技術を導入した場合の被介護者)の場合の可視化]{健常者(介護技術を導入した場合の被介護者)の場合の可視化 \label{elderly_v1}}
\end{center}
\end{figure}

\begin{figure}[htb]
\begin{center}
 \includegraphics[scale=0.5]{figures/elderly_v2.png}
 \caption[頻尿の場合の可視化]{頻尿の場合の可視化 \label{elderly_v2}}
\end{center}
\end{figure}

\begin{figure}[htb]
\begin{center}
 \includegraphics[scale=0.5]{figures/elderly_v3.png}
 \caption[認知症の場合の可視化]{認知症の場合の可視化 \label{elderly_v3}}
\end{center}
\end{figure}

ここでは,介護シミュレーションを可視化するために,これら3種類のエージェントを介護環境に配置する.それぞれが介護アラートを出す条件を,健常者の場合は100ml以上になった時点,頻尿の場合は75ml以上になった時点,認知症の場合は150ml以上で介護アラートを発する設定にし,アラートが発された時に被介護者とランダムにマッチングし,トイレへと向かうよう介護行動を設定した.この時,図\ref{care_alert_v1}のように,被介護者エージェントが介護アラートを出した時に、介護者エージェントが被介護者エージェントの元へ行き,トイレと連れて行く様子を可視化することができた.また,図\ref{alerts_v1}のように,複数の被介護者が同時に介護アラートを出した時は,介護者エージェントが介護すべき最適なエージェントの元へ行き,介護行動を行う様子も可視化することができた.そして図\ref{care_toilet_v1}のように,その介護が終わると同時に,残りの被介護者エージェントの元へと介護行動に向かわせることができた.したがって本シミュレーションは,介護環境を再現するのに必要最低限の機能を有したシミュレーションであると言える.なお,必要最低限の機能とは,被介護者がアラートを出した時点で,介護すべき最適な介護者とマッチングし,その介護者が適切に介護を行うことができることである.

\begin{figure}[htb]
\begin{center}
 \includegraphics[scale=0.5]{figures/care_alert_v1.png}
 \caption[介護行動の可視化]{介護行動の可視化 \label{care_alert_v1}}
\end{center}
\end{figure}

\begin{figure}[htb]
\begin{center}
 \includegraphics[scale=0.5]{figures/alerts_v1.png}
 \caption[複数介護アラートが出た場合の可視化]{複数介護アラートが出た場合の可視化 \label{alerts_v1}}
\end{center}
\end{figure}

\begin{figure}[htb]
\begin{center}
 \includegraphics[scale=0.5]{figures/care_toilet_v1.png}
 \caption[介護行動の可視化]{介護行動の可視化 \label{care_toilet_v1}}
\end{center}
\end{figure}

\thispagestyle{myheadings}
\markboth{_}{_}
\mbox{}\newpage

% 5章
\chapter{数値実験}

研究で新たに実装した歩行者工一ジェントおよび介護シミュレーションの定量的な評価性能を検証するために,シミュレーション実験を行った.
この章では,実験で用いた環境設定と,実験結果をどのような評価指標で判断したか,またその結果と考察についてまとめている.

\section{実験条件}

今回の実験では介護環境として,図\ref{environment}に示すように,15m四方の二次元平面と,排泄場所としてのトイレをその上部に設置した.

\begin{figure}[htb]
\begin{center}
 \includegraphics[scale=0.5]{figures/environment.png}
 \caption[実験環境]{実験環境 \label{environment}}
\end{center}
\end{figure}

この環境の中で,介護者と被介護者の可視化をおこなっていく.今回の実験では,介護における技術を導入した際に,それが介護環境にどのようなインパクトをあたえるのかについて検証を行うことが目的なので,介護者の数は1人,被介護者の数は16人と設定し,比較的大きい施設を対象とした.第4章で述べた健常者,頻尿の被介護者,認知症の被介護者の3種類のエージェントが,介護施設内の自由時間である2時間の間にいかなる回数にわたって排泄介助を行うことが必要か,またその介助は本当に必要であったのかということを確かめる実験をおこなう.3種類の被介護エージェントを\ref{experiment}のように組み合わせた6パターンにおいて,相互作用を確認し,それを本章で示す評価指標で評価する.

\begin{figure}[htb]
\begin{center}
 \includegraphics[scale=0.5]{figures/health_urinate.png}
 \caption[健常者と認知症の場合の可視化]{健常者と認知症の場合の可視化 \label{health_urinate}}
\end{center}
\end{figure}

\begin{figure}[htb]
\begin{center}
 \includegraphics[scale=0.5]{figures/health_frequently_urinate_v1.png}
 \caption[健常者と頻尿の場合の可視化]{健常者と頻尿の場合の可視化 \label{health_frequently_urinate_v1}}
\end{center}
\end{figure}

\begin{figure}[htb]
\begin{center}
 \includegraphics[scale=0.5]{figures/dementia_urinate_v1.png}
 \caption[認知症と頻尿の場合の可視化]{認知症と頻尿の場合の可視化 \label{dementia_urinate_v1}}
\end{center}
\end{figure}

\begin{table}[htb]
  \caption[実験条件]{実験条件}
  \label{experiment}
  \centering
  \begin{tabular}{r|c|c|c|c|c|c}
     & Case1 & Case2 & Case3 & Case4 & Case5 & Case6 \\ \hline
    健常者 & 100% & 0% & 0% & 50% & 50% & 0% \\
    頻尿   & 0% & 100% & 0% & 50% & 0% & 50% \\
    認知症 & 0% & 0% & 100% & 0% & 50% & 50% \\
    \end{tabular}
\end{table}

表\ref{experiment}に示したように,Case1は、健常の被介護者(技術の補助を受けた被介護者)が100%の状態,Case2は頻尿の被介護者が100%の状態,Case3は認知症の被介護者が100%の状態,Case4は、健常の被介護者(技術の補助を受けた被介護者)と頻尿の被介護者が50%ずつの状態,Case5は健常の被介護者(技術の補助を受けた被介護者)と認知症の被介護者が50%ずつの状態,Case6は頻尿の被介護者と認知症の被介護者が50%ずつの状態である.

\section{介護挙動の基本的な検証}

本シミュレータが現実を反映できているのかについて,簡単な検証を行った.介護アラートを出す条件を,健常者の場合は100ml以上になった時点,頻尿の場合は75ml以上になった時点,認知症の場合は150ml以上で介護アラートを発する設定にし,2時間シミュレータを回した.その結果,健常者の場合は,2時間に1回トイレに行くという結果を得ることができた.これは実際のデータと比較しても整合性のある値となった.この結果を表\ref{number_of_urination}に示す.なお,実験では16人の被介護者が存在するので実際は数値を16で割った数字が一人当たりの回数となっている.

\begin{table}[htb]
  \caption[被介護者ごとの排尿回数]{被介護者ごとの排尿回数}
  \label{number_of_urination}
  \centering
  \begin{tabular}{r|c|c|c}
     & 健常者 & 頻尿 & 認知症 \\ \hline
    一回目 & 15 & 21 & 6 \\
    二回目 & 15 & 22 & 5 \\
    三回目 & 16 & 26 & 5 \\
    \end{tabular}
\end{table}

\section{評価指標}

本研究の目的は,疾患のある被介護者,すなわち現状介護者の負担増の原因となっており,被介護者自身も自らの排泄が負担となっているようなケースにおいて,技術の導入を行うことでどれだけの効果が得られるのかを可視化するというものである.そこで,評価指標としては,排泄に行くべきである尿量の状態,あるいは自身が排泄に行きたいと感じている状態から実際に排泄を行うまでの時間を計測し,それを総我慢時間とし,本研究の評価指標とする.

\section{結果および考察}

図\ref{result_v1}に,介護者・被介護者の割合をCase1からCase6までそれぞれ変化させた場合のシミュレーション結果(10回の試行の平均値)を示す.表\ref{relative_error}に,それぞれの具体的な数値とCase1に対する相対誤差を示した.次に表\ref{number_of_care}に,Caseごとの介護回数とCase1に対する相対誤差を示した.

\begin{figure}[htb]
\begin{center}
 \includegraphics[scale=0.5]{figures/result_1.png}
 \caption[実験結果]{実験結果 \label{result_v1}}
\end{center}
\end{figure}

\begin{table}[htb]
  \caption[Caseごとの我慢時間]{Caseごとの我慢時間}
  \label{relative_error}
  \centering
  \begin{tabular}{r|c|c|c|c|c|c}
     & Case1 & Case2 & Case3 & Case4 & Case5 & Case6 \\ \hline
    我慢時間 & 220.1 & 764.0 & 2213.5 & 278.4 & 1285.9 & 1363.7 \\
    相対誤差 & & 2.47 & 9.05 & 0.26 & 4.84 & 5.19 \\
    \end{tabular}
\end{table}


\begin{table}[htb]
  \caption[Caseごとの介護回数]{Caseごとの介護回数}
  \label{number_of_care}
  \centering
  \begin{tabular}{r|c|c|c|c|c|c}
     & case1 & Case2 & Case3 & Case4 & Case5 & Case6 \\ \hline
    介護回数 & 15-16 & 21-26 & 5-6 & 19-21 & 9-10 & 16-17 \\
    相対誤差 & & 0.50 & -0.65 & 0.31 & -0.37 & 0.07 \\
    \end{tabular}
\end{table}

Case2については,過剰介護が問題であると考えられる.健常者の場合と比べ,50%以上も介護士の労働力に悪影響を与えているといえる.頻尿の場合は,単純にトイレに行かないという選択肢を現場で取ることができないので,おむつなどで現場では対応していると考えられる.しかしおむつは,被介護者の自立意識を妨げるツールであるので,その使用については注意が必要である.どのタイミングでおむつをつけることを決め,どの時点でつけることをやめるのか,その意思決定基準はブラックボックスになっている.しかしCase4の結果を見ると,被介護者の我慢時間がCase1との相対誤差が小さいことに加えて,実際の介護回数も相対誤差0.31に留まっている.すなわちシミュレーションによって,高齢者施設がどういった状態の被介護者を受け入れるか,あるいはどの被介護者同士を同じ部屋に入れ,一人の被介護者に対応させるのかといった人事配置について示唆を得ることができることを示している.

Case3の総我慢時間が,もっとも高いものとなっているが,これは被介護者が本来なら排泄に向かうべき尿量であるのにも関わらず,無意識のうちに我慢をしてしまい,介護者にアラートを出した時点ですでにかなりの時間を待ち時間として計測してしまっていることが原因であることが考えられる.逆に言えば,技術を導入することでCase1の状態へと環境を変えることで,被介護者の我慢時間を述べ約2000秒も短縮することができる.

Case5では,数値上は相対誤差がかなり少ないように見えるが,実際は健常者と認知症の被介護者の間の待ち時間の差が大きく,健常者の割合が減ったことで,健常者がアラートを出した際にすぐ介護してもらえたということがあげられる.実際の現場では,被介護者の要介護によって,どの介護者がつくべきかということが事前情報として与えられているため,このような複雑な状況にも対応していけるような環境をつくっているという示唆を得ることができる.今後の検討課題として,そういった事前情報の有無によって,どの介護者と被介護者をマッチングさせるのかということが挙げられる.具体的にどのような事前情報に基づいて意思決定が行われているのか今後の調査課題である.

Case6は我慢時間こそ多いものの,介護回数についてはCase1とそれほど違いが出ていない.しかし,これも頻尿の被介護者に対する介護が多く,認知症の被介護者に対する介護回数が少ないので,合計回数が相殺してCase1と近い値が出てしまったためである.

%
%% 後付け
\backmatter
\thispagestyle{myheadings}
\markboth{_}{_}
\mbox{}\newpage

\chapter{謝辞}

指導教員であり,本研究の機会を与えていただいた,吉村教授に感謝いたします.研究への姿勢や考え方など,右も左もわからなかった私に一から丁寧に教えていただきました.また,就職活動でもお気遣いいただき,本当に感謝しております.研究を通じて先生に教えていただいた,物事を構造的に捉え,仮説を持って取り組む姿勢は,授業では決して得られない貴重な学びになりました.先生から学んだ仮説思考とも言える考え方は,研究の領域に留まらず,人生においてとても重要なものであると確信しております.今回私は,これまで研究室にはなかった医療というフィールドでの研究を0から始めたわけですが,このようなチャレンジングな機会を与えていただいたこと,また研究内容が社会的意義のあるものだと感じたまま研究に取り組めたことは自分の修士課程にとってとても大きなことでした.本研究での経験や学びを忘れず,社会のために生かして参りたいと存じます.ゼミのディスカッションや研究打ち合わせでは,私に合わせて理解のしやすい説明も織り交ぜながら,医療現場について,研究発表について,教えていただきました.先生とのミーティングは,医療業界全体が抱えるマクロ的な課題をはじめとして,「私の母親の例だと〜」といったようにミクロ的な物事の捉えかた,縦横無尽に課題を探索していく思考の柔軟さに圧倒されることが多く,非常に学びの多い時間でした.とても忙しいはずなのに何冊も本を読み,貪欲に知識を蓄えていく姿勢は,私も今後死ぬまで持ち続けていきたいと心から感じております.これからは,吉村研究室の名に恥じない,社会に価値を提供できる人間になり,吉村研究室の卒業生として世界に名を馳せるよう精進いたします.また,いつも単位のことを気にかけてくださり,本当に感謝してもしきれません.約二年間,大変お世話になりました.ありがとうございました.

%藤井先生

%内田さん

%阿部さん

%研究室の皆様

%同期

%井上さん
 % 謝辞
\thispagestyle{myheadings}
\markboth{_}{_}
\mbox{}\newpage

\begin{thebibliography}{99}
 \bibitem{ex_kousei_v1} 厚生労働省, 平成28年版厚生労働白書(2016)
 \bibitem{nursing_management} 田宮菜々子:高齢者にもとづく高齢者施設ケア.33/35(2010)
 \bibitem{2025_problem} 厚生労働省,医療と介護を取り巻く現状 (2016)
 %\bibitem{2025_problem}https://www.mhlw.go.jp/file/05-Shingikai-12404000-Hokenkyoku-Iryouka/0000167844.pdf
 \bibitem{population_GDP_relation} 三菱リサーチコンサルティング,日本経済の中期見通し
 %\bibitem{population_GDP_relation}https://www.murc.jp/wp-content/uploads/2019/04/medium_1904.pdf
 \bibitem{social_security} 内閣府,社会保障給付費の推移等
 %\bibitem{social_security} https://www5.cao.go.jp/keizai-shimon/kaigi/special/2030tf/281020/shiryou1_2.pdf
 \bibitem{lack_facility_1} みずほ情報総研株式会社,特別養護老人ホームの開設状況に関する調査研究(2016)
 %\bibitem{lack_facility_1} https://www.mhlw.go.jp/file/06-Seisakujouhou-12300000-Roukenkyoku/63_mizuho_1.pdf
 \bibitem{lack_facility_2} 大和総研,超高齢社会における介護問題
 %\bibitem{lack_facility_2} https://www.dir.co.jp/report/research/policy-analysis/social-securities/20140509_008508.pdf
 \bibitem{lack_} 厚生労働省,介護老人福祉施設(2017)
 %\bibitem{lack_} https://www.mhlw.go.jp/file/05-Shingikai-12601000-Seisakutoukatsukan-Sanjikanshitsu_Shakaihoshoutantou/0000171814.pdf
 \bibitem{lack_nurse_1} 厚生労働省,第7期介護保険事業計画に基づく介護人材の必要数について (2018)
 %\bibitem{lack_nurse_1}https://www.mhlw.go.jp/stf/houdou/0000207323.html
 \bibitem{nurse_solution} 厚生労働省,福祉・介護人材確保対策等について(2018)
 %\bibitem{nurse_solution} https://www.mhlw.go.jp/topics/2018/01/dl/tp0115-s01-01-05.pdf
 \bibitem{turnover_rate} 厚生労働省,平成29年上半期雇用動向調査結果の概況(2018)
 %\bibitem{turnover_rate}https://www.mhlw.go.jp/content/12602000/000482541.pdf
 \bibitem{care_robots} 厚生労働省,介護ロボットの開発と普及のための取り組み(2019)
 %\bibitem{care_robots} http://www.techno-aids.or.jp/robot/file01/02shiryo.pdf
 \bibitem{SFM} D.HELBING, P.MOLNAR:Social force model for pedestrian dynamics, Physical Review E 41, 4282,(1995)
 \bibitem{ex_pedestrian_simulation_1} 岡崎甚幸:建築空間における歩行のためのシミュレーションモデルの研究その1 磁気モデルの応用による歩行モデル,目本建築学会論文服告集,2S3,111/ll7(1979)
 \bibitem{ex_pedestrian_simulation_2} Kurdi Teknomo、Groria P.Gerilla:Sensitivity Analysis And Validation of a Multi−Agents Pedestrian Model,Journal of the Eastern Asia Society for Transportation Studies,6,198/213(2005)
 \bibitem{ex_personal_space} 渋谷昌三:パーソナル・スペースの形態に閏する考察,山梨医科大学紀要,2,41/49(1985)
 \bibitem{micturition} 安藤正夫:高齢者における排尿障害の実態について,日本泌尿器科学会誌,82,560/564(1991)
 % \bibitem{mates} MATESの引用
\end{thebibliography}
 % 参考文献
%
\end{document}
