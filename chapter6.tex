% 6章
\chapter{結論}

本研究では,介護支援施設の主体(ここでは研究の第一段階として介護者,被介護者のみに着目する)を,能動的に自分自身で情報を集めて行動する主体性,すなわち自律性を持つ知的エージェントとしてモデル化し,多数の知的工一ジェントを仮想的な介護環境上で相互作用させることによって,介護行動をシミュレーションする知的マルチエージェントモデルに基づくシミュレータを構築した.本論文では,はじめに,本シミュレータに組み込まれた知的マルチエージェントモデルの理論及び実装法について述べた.次に,被介護者の状態によっていくつかの異なる環境を構成し、医療技術を導入した際のインパクトを被介護者の我慢時間という数値を用いて定量的に評価することができた.また,被介護者の種類を変化させ,複数のシミュレーションを行うことで,介護施設の人事配置にも示唆を与えることができる可能性について,言及することができた.

今後の課題としては,絶対値レベルでの定量的な現状再現性については,信頼できる介護施設データが足りておらず,理論値を元に実測値と比較することでモデルの有用性について検証することである.また、介護者と被介護者の最適なマッチングについて,医療現場ではどのような事前情報に基づき行われているのかについて調査を行い,モデルに組み込んでいくことが求められる.本研究では被介護者の我慢時間という数値で評価を行なったが,被介護者の業務負担を定量的に評価できる指標の開発も重要である.
