% 4章
\chapter{結論}

\section{本研究のまとめ}

本研究では,介護支援施設の主体(ここでは研究の第一段階として介護者,被介護者のみに着目する)を,能動的に自分自身で情報を集めて行動する主体性,すなわち自律性を持つ知的エージェントとしてモデル化し,多数の知的工一ジェントを仮想的な遭路環境上で相互作用させることによって,介護環境をシミュレーションする,知的マルチエージェントモデルに基づくシミュレータを構築した.本論文では,はじめに,本シミュレータに組み込まれた知的マルチエージェントモデルの理論及び実装法について述べた.次に,被介護者の状態によっていくつかの異なる環境を構成し、将来医療技術が発達して行くとどれほどのインパクトをもたらすのかについて,かなり艮好な数値を示すことを確認した.

\section{今後の展望}
本シミュレータについて,絶対値レベルでの定量的な現状再現性については,信頼できる介護施設データが足りておらず,理論値を元に実測値と比較することでモデルの有用性について検証して行く必要がある.

% 介護士の業務量について定量的に評価できる指標の導入も行いたい.
